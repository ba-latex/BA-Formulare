%! Code-Integration im Dokument
%? Inklusive Erzeugung eines Custom-Enviroments für Programmcodes
\usepackage{minted}

% ! Anpassung der Minted-Umgebung, um Abstände zu vereinheitlichen und die Optik zu verbessern
\let\oldminted\minted
\let\oldendminted\endminted
\def\minted{\begingroup \vspace{-0.3cm} \oldminted}
\def\endminted{\oldendminted \vspace{-0.5cm} \endgroup}

\xapptocmd{\inputminted}{\vspace{-0.5cm}}{}{}

%Zeilennummern neu definieren, um Warnungen zu vermeiden und modern anmutende Zahlen zu nutzen
\renewcommand{\theFancyVerbLine}{\scriptsize{\arabic{FancyVerbLine}}}

%Standardformatierung für minted-Umgebung erstellen
\setminted{
    tabsize=2,
    breaklines,
    autogobble,
    fontfamily=courier,
    linenos,
    %! see https://pygments.org/demo/#try, other nice options: paraiso-light, solarized-light, rainbow_dash, gruvbox-light, stata, tango
    style=emacs
}
%Keine Zeilennnummer, wenn der einzeilige \mint-Befehl genutzt wird
\xpretocmd{\mint}{\setminted{linenos=false}}{}{}
\xpretocmd{\minted}{\setminted{linenos=true}}{}{}

\DeclareNewTOC[
  type=code,                           % Name der Umgebung
  types=codes,                         % Erweiterung (\listofschemes)
  float,                               % soll gleiten
  floatpos=H,
  tocentryentrynumberformat=\bfseries, % voreingestellte Gleitparameter
  name=Code,                           % Name in Überschriften
  listname={Programmcodeverzeichnis},  % Listenname
  % counterwithin=chapter
]{loc}
\setuptoc{loc}{totoc}

\renewcommand*{\codeformat}{%
  \codename~\thecode%
  \autodot{\space\space\space}
}
\BeforeStartingTOC[loc]{\renewcommand*\autodot{\space\space\space\space}}
