%  Eidesstattliche Erklärung
%! Dies ist eine zur Nutzung mit LaTeX angepasste Version der in Anhang 6 der Hinweise zur Anfertigung
%! wissenschaftlicher Arbeiten an der Staatlichen Studienakademie Glauchau vorgegebenen Zustimmung
%! zur Plagiatsprüfung für 3 Autoren.
\cleardoublepage
\section{Ehrenwörtliche Erklärung}
    \vspace*{1cm}
    \begin{center}
        \huge\textbf{Ehrenwörtliche Erklärung}\\
    \end{center}
    \vspace*{1cm}
    \normalsize
    Wir erklären hiermit ehrenwörtlich,

    \begin{enumerate}
        \vspace{1cm}
        \item dass wir unsere Belegarbeit mit dem Thema:\\

        \textbf{\TextField[width=\columnwidth,multiline=true, height=2cm, name=Thema]{} }\\

        ohne fremde Hilfe angefertigt haben,
        \item dass wir die Übernahme wörtlicher Zitate aus der Literatur sowie die\\
        Verwendung der Gedanken anderer Autoren an den entsprechenden\\
        Stellen innerhalb der Arbeit gekennzeichnet haben und
        \item dass wir unsere Belegarbeit bei keiner anderen Prüfung vorgelegt haben.\\[1,5cm]
    \end{enumerate}
    Wir sind uns bewusst, dass eine falsche Erklärung rechtliche Folgen haben wird.\\[1,5cm]

   \rule{0.35\columnwidth}{0.4pt}\hspace{0.05\columnwidth}\rule{0.55\columnwidth}{0.4pt}\\
    Ort, Datum\hspace{0.27\columnwidth}Unterschriften


    \newpage
\section{Zustimmung Plagiatsprüfung}

    \vspace*{2mm}

    \begin{minipage}{0.5\columnwidth}
        \includesvg[width=\columnwidth]{ba-gc-logo}
    \end{minipage}
    \begin{minipage}{0.45\columnwidth}
        \begin{flushright}
            {\small nach 4BA-F.219\\}
        \end{flushright}
    \end{minipage}
    \vspace*{2mm}

    \begin{center}
        \textbf{\huge{Erklärung zur Prüfung wissenschaftlicher Arbeiten}}
    \end{center}

    Die Bewertung wissenschaftlicher Arbeiten erfordert die Prüfung auf Plagiate. Die hierzu von der Staatlichen Studienakademie Glauchau eingesetzte Prüfungskommission nutzt sowohl eigene Software als auch diesbezügliche Leistungen von Drittanbietern. Dies erfolgt gemäß \href{https://www.revosax.sachsen.de/vorschrift/1672-Saechsisches-Datenschutzgesetz#p7}{§ 7 des Gesetzes zum Schutz der informationellen Selbstbestimmung im Freistaat Sachsen (Sächsisches Datenschutzgesetz - SächsDSG)} vom 25. August 2003 (Rechtsbereinigt mit Stand vom 31. Juli 2011) im Sinne einer Datenverarbeitung im Auftrag.

    Die Studierenden bevollmächtigen die Mitglieder der Prüfungskommission hiermit zur Inanspruchnahme o. g. Dienste. In begründeten Ausnahmefällen kann der Datenschutzbeauftragte der Berufsakademie Sachsen sowohl von den Verfassern der wissenschaftlichen Arbeit als auch von der Prüfungskommission in den Entscheidungsprozess einbezogen werden.

    \arrayrulewidth=0.5pt

    \begin{table}[H]
        \centering
        \begin{tabularx}{\columnwidth}{|X|X|}
            \hline
            Namen:            & \TextField[width=7cm, name=Namen, multiline=true, height=3cm]{} \\
            \hline
            Matrikelnummern:  & \TextField[width=7cm, name=Matrikelnummern, multiline=true, height=2cm]{} \\
            \hline
            Studiengang:      & \TextField[width=7cm, name=Studiengang]{}\\
            \hline
            Titel der Arbeit: & \TextField[width=7cm, name=Thema, multiline=true, height=2cm]{}\\
            \hline
            Datum:            & \TextField[width=7cm, name=Namen]{}\\
            \hline
            Unterschriften:   & \\
                              & \\
                              & \\
                              & \\
                              & \\
                              & \\
            \hline
        \end{tabularx}
    \end{table}
    \vspace{3cm}
    
    \vfill
